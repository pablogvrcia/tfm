\chapter{Implementation Details}
\label{appendix:implementation_details}

This appendix provides specific implementation details, code snippets, and configuration files used in the thesis. These technical specifics support the methodology described in Chapter \ref{ch:methodology}.
\section{Descriptor Files}
\label{sec:appendix_descriptors}

The descriptor file provides rich descriptions for PASCAL VOC 2012 classes. Below is the content used for the key classes discussed.

\textbf{Background class:}
\begin{verbatim}
sky, wall, tree, wood, grass, road, sea, river, mountain,
sands, desk, bed, building, cloud, lamp, door, window,
wardrobe, ceiling, shelf, curtain, stair, floor, hill,
rail, fence
\end{verbatim}

\textbf{Person class:}
\begin{verbatim}
person, person in shirt, person in jeans, person in dress,
person in sweater, person in skirt, person in jacket
\end{verbatim}

\textbf{Compound classes:}
\begin{verbatim}
diningtable, table
television monitor, tv monitor, monitor, television, screen
\end{verbatim}

\section{Template Strategy Implementation}
\label{sec:appendix_templates}

Table~\ref{tab:imagenet80_templates} provides the complete set of 80 ImageNet templates used as the baseline strategy. These templates were designed for ImageNet classification and provide comprehensive coverage of visual variations (viewpoint, lighting, rendering style, etc.).

\begin{table}[p]
\centering
\tiny
\begin{tabular}{rp{6.8cm}rp{6.8cm}}
\hline
\multicolumn{4}{c}{\textbf{ImageNet-80 Templates (OpenAI Baseline)}} \\
\hline
\textbf{\#} & \textbf{Template} & \textbf{\#} & \textbf{Template} \\
\hline
1 & a bad photo of a \{c\}. & 41 & the \{c\} in a video game. \\
2 & a photo of many \{c\}. & 42 & a sketch of a \{c\}. \\
3 & a sculpture of a \{c\}. & 43 & a doodle of the \{c\}. \\
4 & a photo of the hard to see \{c\}. & 44 & a origami \{c\}. \\
5 & a low resolution photo of the \{c\}. & 45 & a low resolution photo of a \{c\}. \\
6 & a rendering of a \{c\}. & 46 & the toy \{c\}. \\
7 & graffiti of a \{c\}. & 47 & a rendition of the \{c\}. \\
8 & a bad photo of the \{c\}. & 48 & a photo of the clean \{c\}. \\
9 & a cropped photo of the \{c\}. & 49 & a photo of a large \{c\}. \\
10 & a tattoo of a \{c\}. & 50 & a rendition of a \{c\}. \\
11 & the embroidered \{c\}. & 51 & a photo of a nice \{c\}. \\
12 & a photo of a hard to see \{c\}. & 52 & a photo of a weird \{c\}. \\
13 & a bright photo of a \{c\}. & 53 & a blurry photo of a \{c\}. \\
14 & a photo of a clean \{c\}. & 54 & a cartoon \{c\}. \\
15 & a photo of a dirty \{c\}. & 55 & art of a \{c\}. \\
16 & a dark photo of the \{c\}. & 56 & a sketch of the \{c\}. \\
17 & a drawing of a \{c\}. & 57 & a embroidered \{c\}. \\
18 & a photo of my \{c\}. & 58 & a pixelated photo of a \{c\}. \\
19 & the plastic \{c\}. & 59 & itap of the \{c\}. \\
20 & a photo of the cool \{c\}. & 60 & a jpeg corrupted photo of the \{c\}. \\
21 & a close-up photo of a \{c\}. & 61 & a good photo of a \{c\}. \\
22 & a black and white photo of the \{c\}. & 62 & a plushie \{c\}. \\
23 & a painting of the \{c\}. & 63 & a photo of the nice \{c\}. \\
24 & a painting of a \{c\}. & 64 & a photo of the small \{c\}. \\
25 & a pixelated photo of the \{c\}. & 65 & a photo of the weird \{c\}. \\
26 & a sculpture of the \{c\}. & 66 & the cartoon \{c\}. \\
27 & a bright photo of the \{c\}. & 67 & art of the \{c\}. \\
28 & a cropped photo of a \{c\}. & 68 & a drawing of the \{c\}. \\
29 & a plastic \{c\}. & 69 & a photo of the large \{c\}. \\
30 & a photo of the dirty \{c\}. & 70 & a black and white photo of a \{c\}. \\
31 & a jpeg corrupted photo of a \{c\}. & 71 & the plushie \{c\}. \\
32 & a blurry photo of the \{c\}. & 72 & a dark photo of a \{c\}. \\
33 & a photo of the \{c\}. & 73 & itap of a \{c\}. \\
34 & a good photo of the \{c\}. & 74 & graffiti of the \{c\}. \\
35 & a rendering of the \{c\}. & 75 & a toy \{c\}. \\
36 & a \{c\} in a video game. & 76 & itap of my \{c\}. \\
37 & a photo of one \{c\}. & 77 & a photo of a cool \{c\}. \\
38 & a doodle of a \{c\}. & 78 & a photo of a small \{c\}. \\
39 & a close-up photo of the \{c\}. & 79 & a tattoo of the \{c\}. \\
40 & a photo of a \{c\}. & 80 & \\
\hline
\end{tabular}
\caption{Complete ImageNet-80 template set.}
\label{tab:imagenet80_templates}
\end{table}

\subsection{Alternative Template Sets}

\subsubsection{Top-7 Dense Prediction Templates}

Based on PixelCLIP (ECCV 2024), these 7 templates were identified:

\begin{enumerate}
    \item \texttt{a photo of a \{c\}.} --- General object recognition
    \item \texttt{a \{c\} in the scene.} --- Spatial context (critical for segmentation)
    \item \texttt{the \{c\}.} --- Definite article (helps with stuff classes)
    \item \texttt{a close-up photo of a \{c\}.} --- Detail and texture focus
    \item \texttt{a photo of the large \{c\}.} --- Size variation (large objects)
    \item \texttt{a photo of the small \{c\}.} --- Size variation (small objects)
    \item \texttt{one \{c\}.} --- Instance awareness (countability)
\end{enumerate}

These templates emphasize spatial context (\textit{"in the scene"}) and size variation, which are particularly important for dense prediction tasks compared to image-level classification.

\subsubsection{Top-3 Ultra-Fast Templates}

For testing if minimal templates can still yield strong results, the following 3 templates were selected from the Top-7 set:

\begin{enumerate}
    \item \texttt{a photo of a \{c\}.} --- General recognition
    \item \texttt{a \{c\} in the scene.} --- Spatial context
    \item \texttt{the \{c\}.} --- Definite article
\end{enumerate}

This minimal set retains the most critical templates for segmentation: generic recognition, spatial context, and definite reference.

\subsubsection{Adaptive Class-Type Templates}

Following DenseCLIP (CVPR 2022) and CLIP-DIY (CVPR 2024), different templates are selected based on class type. Classes are categorized as \textbf{stuff} (amorphous regions: sky, grass, road, wall) or \textbf{things} (countable objects: person, car, chair, bottle).

\textbf{Stuff templates} (5 templates) emphasize continuity and lack of boundaries:
\begin{enumerate}
    \item \texttt{the \{c\}.} --- Definite article (no counting)
    \item \texttt{a photo of \{c\}.} --- No article (mass noun)
    \item \texttt{\{c\} in the scene.} --- Background context
    \item \texttt{\{c\} in the background.} --- Explicit background
    \item \texttt{a region of \{c\}.} --- Spatial extent
\end{enumerate}

\textbf{Thing templates} (5 templates) emphasize individuality and object-ness:
\begin{enumerate}
    \item \texttt{a photo of a \{c\}.} --- Indefinite article (countable)
    \item \texttt{one \{c\}.} --- Explicit counting
    \item \texttt{a \{c\} in the scene.} --- Object in context
    \item \texttt{the \{c\} in the image.} --- Definite object
    \item \texttt{a photo of the \{c\}.} --- Object focus
\end{enumerate}

The adaptive strategy applies stuff templates to classes like sky, grass, road, and wall, while applying thing templates to countable objects like person, car, and bottle. PASCAL VOC 2012 contains predominantly thing classes (19 of 20 foreground classes), explaining why the adaptive strategy provides smaller gains compared to datasets with more balanced stuff/thing distribution.
