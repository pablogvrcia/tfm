\chapter{Implementation Details}
\label{appendix:implementation_details}

This appendix provides specific implementation details, code snippets, and configuration files used in the thesis. These technical specifics support the methodology described in Chapter \ref{ch:methodology}.

\section{Descriptor Files}
\label{sec:appendix_descriptors}

The descriptor file \texttt{configs/cls\_voc21.txt} provides rich descriptions for PASCAL VOC 2012 classes. Below is the content used for the key classes discussed in Section \ref{sec:descriptor_files}.

\textbf{Background class (line 1):}
\begin{verbatim}
sky, wall, tree, wood, grass, road, sea, river, mountain,
sands, desk, bed, building, cloud, lamp, door, window,
wardrobe, ceiling, shelf, curtain, stair, floor, hill,
rail, fence
\end{verbatim}

\textbf{Person class (line 16):}
\begin{verbatim}
person, person in shirt, person in jeans, person in dress,
person in sweater, person in skirt, person in jacket
\end{verbatim}

\textbf{Compound classes:}
\begin{verbatim}
table                    (line 12: diningtable)
television monitor, tv monitor, monitor, television, screen
                         (line 21: tvmonitor)
\end{verbatim}

\section{Template Strategy Implementation}
\label{sec:appendix_templates}

The template strategies discussed in Section \ref{sec:template_strategies} are implemented in Python. The \texttt{Top-7 Dense Prediction} strategy is defined as follows:

\begin{verbatim}
top7_dense_templates = [
    lambda c: f'a photo of a {c}.',
    lambda c: f'a {c} in the scene.',
    lambda c: f'the {c}.',
    lambda c: f'a close-up photo of a {c}.',
    lambda c: f'a photo of the large {c}.',
    lambda c: f'a photo of the small {c}.',
    lambda c: f'one {c}.',
]
\end{verbatim}

\section{Computational Optimizations}
\label{sec:appendix_optimizations}

This section provides the PyTorch implementation details for the optimizations described in Section \ref{sec:computational_optimizations}.

\subsection{Mixed Precision Inference (FP16)}

The implementation uses \texttt{torch.cuda.amp.autocast} to enable mixed precision:

\begin{verbatim}
with torch.cuda.amp.autocast(enabled=use_fp16):
    # CLIP vision encoder forward pass
    image_features = clip_model.encode_image(images)
    # Text encoder forward pass
    text_features = clip_model.encode_text(text_tokens)
    # Similarity computation
    similarity = image_features @ text_features.T
\end{verbatim}

\subsection{Just-In-Time Compilation}

PyTorch 2.0 compilation is applied to CLIP encoders:

\begin{verbatim}
if use_compile:
    clip_model.visual = torch.compile(
        clip_model.visual,
        mode="reduce-overhead"
    )
    clip_model.transformer = torch.compile(
        clip_model.transformer,
        mode="reduce-overhead"
    )
\end{verbatim}

\subsection{Batched SAM2 Prompting}

Batching multiple prompts for efficient SAM2 inference:

\begin{verbatim}
# Standard approach (sequential, slow)
masks = []
for point in prompt_points:
    mask = sam2_predictor.predict(point)
    masks.append(mask)

# Batched approach (parallel, fast)
batch_size = 32
for i in range(0, len(prompt_points), batch_size):
    batch = prompt_points[i:i+batch_size]
    batch_masks = sam2_predictor.predict_batch(batch)
    masks.extend(batch_masks)
\end{verbatim}

\section{VACE Integration Details}
\label{sec:appendix_vace_implementation}

\subsection{Video Encoding with FFmpeg}

The binary mask frames are encoded into an MP4 video using FFmpeg with H.264 codec to ensure compatibility with VACE:

\begin{verbatim}
ffmpeg -framerate <fps> -i frame_%06d.png
       -c:v libx264 -pix_fmt yuv420p -crf 23
       output_mask.mp4
\end{verbatim}

\subsection{VACE Inference Arguments}

The VACE inference is invoked with specific arguments to handle path management and model configuration:

\begin{verbatim}
# Change to VACE directory for correct imports
original_dir = os.getcwd()
vace_path = os.path.join(os.path.dirname(__file__), 'VACE')
os.chdir(vace_path)

from vace.vace_wan_inference import main as vace_main

# Prepare arguments with absolute paths
vace_args = {
    'model_name': 'vace-1.3B',
    'ckpt_dir': 'models/Wan2.1-VACE-1.3B/',
    'src_video': abs_preview_video_path,
    'src_mask': abs_mask_video_path,
    'prompt': edit_prompt,
    'save_file': abs_output_path,
    ...
}

result = vace_main(vace_args)
os.chdir(original_dir)
\end{verbatim}
